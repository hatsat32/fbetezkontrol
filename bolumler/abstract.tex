\theabstract{
Since its introduction in 2013, Docker has become a very popular standard in the IT industry. The biggest reason for this is that Docker provides various advantages over traditional virtualisation technologies. Thanks to these advantages, Docker images today form the basis of cloud technologies and applications used in cloud technologies.

Although they have many uses and advantages, Docker images contain various security risks. Security vulnerabilities in unpatched Docker images pose a serious threat to organisations and can lead to financial losses and disclosure of private data. In addition, these security risks can be worsened when malicious container images are used.

Since Docker images form the basis of containerised applications, structural weaknesses within these images can make the entire installation or application vulnerable to attack. Moreover, due to the dynamic nature of software development, the risk environment is constantly changing. New vulnerabilities are discovered every day and security updates are released to patch these vulnerabilities. This process involves being informed about newly discovered vulnerabilities and scanning images with effective tools against vulnerabilities.

Two important studies were carried out within the scope of the thesis. The first study is to measure the effectiveness of security scanning tools used in the scanning of Docker images and to compare their effectiveness. In this study, measurements were made using the evaluation metrics used in the literature as well as two new metrics proposed in the thesis. In the second study, temporal improvements between versions of Docker images were investigated. For these purposes, 439 most used docker images from Docker Hub were analysed. These images were scanned and evaluated with Trivy, Grype and Snyk tools in the first study. In the second study, three different versions were selected for each image at least six months apart and a total of 1092 scans were performed with the same tools. In this study, the results in varying time periods were analysed and analyses were made regarding the vulnerability change of Docker images.
}

\keywords{Docker, Docker Security, Docker Images, Security, Vulnerability Detection.}
